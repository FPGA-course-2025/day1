\documentclass{beamer}
\setbeamertemplate{navigation symbols}{}

\usepackage[english]{babel}
\usepackage{array}
\usepackage{listings}
\usepackage{color}
\usepackage{hyperref}
\usepackage{trace}
\usepackage[scriptsize]{caption}

\usepackage{tikz}
\usetikzlibrary{arrows,shapes,positioning,calc}

\usetheme{Boadilla}

%\setbeamertemplate{footline}
%{
%  \leavevmode
%  \hbox{
%    \begin{beamercolorbox}[wd=.5\paperwidth,ht=2.5ex,dp=1.125ex,leftskip=.3cm,rightskip=.3cm plus1fil]{author in head/foot}
%      \usebeamerfont{author in head/foot}\insertshortauthor \hfill \inserttitle
%    \end{beamercolorbox}
%    \begin{beamercolorbox}[wd=.48\paperwidth,ht=2.5ex,dp=1.125ex,leftskip=.3cm,rightskip=.3cm plus1fil]{title in head/foot}
%      \usebeamerfont{title in head/foot}\insertshortdate \hfill \insertframenumber{}/\inserttotalframenumber
%    \end{beamercolorbox}
%  }
%  \vskip0pt
%}

\definecolor{lgray}{rgb}{0.8,0.8,0.8}
\definecolor{dgreen}{rgb}{0.0,0.8,0.0}

\lstset{breakatwhitespace,
language=C,
backgroundcolor=\color{lgray},
%columns=fullflexible,
keepspaces,
breaklines,
tabsize=3, 
frame=shadowbox,
basicstyle=\ttfamily\fontsize{5}{6}\selectfont,
showstringspaces=false,
keywordstyle=\color{blue},
commentstyle=\color{dgreen},
extendedchars=true}

\setbeamercovered{transparent}

\begin{document}
\title[\today]{Git: Main Concepts}
\author{Mirko Mariotti}
\date{\today}

\begin{frame}
{
    \includegraphics[width=0.2\textwidth]{mi.png}  
    \hfill
    \includegraphics[width=0.7\textwidth]{iscs.png}
}
\titlepage
\end{frame}

% \begin{frame}\frametitle{Contents}
% \fontsize{6}{7.2}\selectfont
% \tableofcontents
% \end{frame} 

\section{Introduction}

\subsection{What is Git ?}

\begin{frame}\frametitle{What is Git ?}
\begin{itemize}
	\item Git is a distributed version control system.
	\item It is used to track changes in source code during software development.
	\item It allows multiple developers to work on a project simultaneously without conflicts.
	\item Git is designed to handle everything from small to very large projects with speed and efficiency.
\end{itemize}
\end{frame}

\subsection{Basic Concepts}

\begin{frame}\frametitle{Basic Concepts}\framesubtitle{Setting up a working directory: Repository, Init, Clone}
\begin{itemize}
	\item \textbf{Repository (repo)}: A directory that contains all the files and history of a project.
	\vspace{0.2cm}
	\item \textbf{Init}: The command to create a new Git repository.
	\vspace{0.2cm}
	\item \textbf{Clone}: A copy of a repository that is stored on your local machine.
\end{itemize}

\begin{alertblock}{Initialize a repository}
cd some-empty-directory \\
git init
\end{alertblock}

\vspace{0.5cm}

\begin{alertblock}{Clone a repository}
git clone https://github.com/FPGA-course-2025/day1.git
\end{alertblock}

\end{frame}

\begin{frame}\frametitle{Basic Concepts}\framesubtitle{Working inside a repository: Add, Commit, Status}
\begin{itemize}
	\item \textbf{Add}: Stages changes in the working directory for the next commit.
	\vspace{0.2cm}
	\item \textbf{Commit}: Saves the staged changes to the repository with a message describing the changes.
	\vspace{0.2cm}
	\item \textbf{Status}: Displays the state of the working directory and staging area.
\end{itemize}

\begin{alertblock}{Stage changes}
git add file1 file2 ...
\end{alertblock}

\begin{alertblock}{Commit changes}
git commit -m "Commit message"
\end{alertblock}

\begin{alertblock}{Check status}
git status
\end{alertblock}

\end{frame}

\begin{frame}\frametitle{Basic Concepts}\framesubtitle{Working with branches: Branch, Checkout, Merge}
\begin{itemize}
	\item \textbf{Branch}: A separate line of development in a repository.
	\vspace{0.2cm}
	\item \textbf{Checkout}: Switches to a different branch or	 restores working tree files.
	\vspace{0.2cm}
	\item \textbf{Merge}: Combines changes from one branch into another.
\end{itemize}

\begin{alertblock}{Create a new branch}
git branch new-branch
\end{alertblock}

\begin{alertblock}{Switch to a branch}
git checkout new-branch
\end{alertblock}

\begin{alertblock}{Merge branches}
git merge branch-to-merge
\end{alertblock}

\end{frame}

\begin{frame}\frametitle{Basic Concepts}\framesubtitle{Remote repositories: Remote, Push, Pull}
\begin{itemize}
	\item \textbf{Remote}: A version of the repository that is hosted on a server.
	\vspace{0.2cm}
	\item \textbf{Push}: Uploads local changes to a remote repository.
	\vspace{0.2cm}
	\item \textbf{Pull}: Fetches changes from a remote repository and merges them into the local repository.
\end{itemize}

\begin{alertblock}{Add a remote repository}
git remote add origin remote-repo-url (automatically created when cloning)
\end{alertblock}

\begin{alertblock}{Push changes to remote}
git push origin branch-name (or git push)
\end{alertblock}

\begin{alertblock}{Pull changes from remote}
git pull origin branch-name (or git pull)
\end{alertblock}

\end{frame}

\begin{frame}\frametitle{github}\framesubtitle{GitHub: A popular platform for hosting Git repositories}
\begin{itemize}
	\item GitHub is a web-based platform that provides hosting for Git repositories.
	\item It offers features like issue tracking, pull requests, and collaboration tools.
	\item GitHub allows developers to share their code, collaborate on projects, and contribute to open-source software.
	\item It provides a user-friendly interface for managing repositories and viewing commit history.
\end{itemize}

\begin{alertblock}{GitHub Organization for this course}
\begin{center}
\url{https://github.com/FPGA-course-2025}
\end{center}
\end{alertblock}
\end{frame}

\end{document}
